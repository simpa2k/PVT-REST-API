\documentclass{article}

\usepackage[utf8]{inputenc}
\usepackage{listings}

\title{API-dokumentation för doSTHLMs REST-API}
\author{Simon Olofsson}
\date{}

\begin{document}

\lstset{showspaces=false,showstringspaces=false,literate={Ä}{{\"A}}1{Ö}{{\"O}}1}

\maketitle

\newpage

\tableofcontents

\newpage

\section*{Endpoints}
\addcontentsline{toc}{section}{Endpoints}

\subsection*{/facebook/login}
\addcontentsline{toc}{subsection}{/facebook/login}

\subsubsection*{POST}
\addcontentsline{toc}{subsubsection}{POST}

Loggar in en användare via facebook. Förväntar sig en HTTP-body med följande utseende:

\begin{lstlisting}
{
	emailAddress: ANVÄNDARENS EMAILADRESS,
	password: ANVÄNDARENS LÖSENORD
}
\end{lstlisting}

\noindent Returnerar en autentiseringstoken som kan kommas åt antingen via en cookie som sätts
eller själva response-bodyn, som har formatet:

\begin{lstlisting}
{
	authToken: AUTENTISERINGSTOKEN
}
\end{lstlisting}

\noindent Förutsatt att vi lagrar svaret i en variabel kallad \textit{response} kan alltså denna token 
kommas åt som \textit{response.authToken}.

\subsection*{/swipingsessions}
\addcontentsline{toc}{subsection}{/swipingsessions}

\subsubsection*{GET}
\addcontentsline{toc}{subsubsection}{GET}

Hämtar alla swipingsessioner där de angivna användarna deltagit.

\paragraph{Parametrar:}

\begin{itemize}
	
	\item emailAddresses -- en array i json-format innehållandes en serie emailadresser. Matchningen kommer att göras på alla
		emailadresser som skickas, med andra ord kommer bara swipingsessioner där alla användare kopplade till emailaddresserna
		deltagit att hämtas. Matchningen är också exakt i det att den inte heller kommer att göras på de swipingsessioner där
		några andra användare, utöver dessa, deltagit.

\end{itemize}

\paragraph{Exempel:} /swipingsessions?emailAddresses=["user1@example.com","user2@example.com"] 

\subsubsection*{POST}
\addcontentsline{toc}{subsubsection}{POST}

Skapar en ny swipingsession med alla angivna användare som deltagare. Returnerar den nya swipingsessionens id samt en lista med aktiviteter.

\paragraph{Parametrar:}

\begin{itemize}		

	\item emailAddresses -- en array i json-format innehållandes en serie emailaddresser.			

\end{itemize}		

\paragraph{Exempel:} /swipingsessions?emailAddresses=["user1@example.com","user2@example.com"] 

\subsubsection*{PUT}
\addcontentsline{toc}{subsubsection}{PUT}

Uppdaterar en swipingsession med valda aktiviteter för den angivna användaren.

\paragraph{Parametrar:}

\begin{itemize}

	\item swipingSessionId -- id:t för den swipingsession som ska uppdateras. Fås vid skapandet av swipingsessionen eller genom en
		GET-request till denna endpoint.
	\item email -- den emailadress associerad med den användare som just valt aktiviteter.
	\item activities -- en array i json-format innehållandes namnet på de aktiviteter som valts.

\end{itemize}

\paragraph{Exempel:} /swipingsessions?swipingSessionId=1\&email=user1@example.com\&activities=["Moderna Museet","Medelhavsmuseet"]

\end{document}
