\documentclass{article}

\usepackage[utf8]{inputenc}
\usepackage[swedish]{babel}
\usepackage{listings}

\title{API-dokumentation för doSTHLMs REST-API}
\author{Simon Olofsson}
\date{}

\begin{document}

\lstset{showspaces=false,showstringspaces=false,literate={Ä}{{\"A}}1{Ö}{{\"O}}1}

\maketitle

\newpage

\tableofcontents

\newpage

\section*{Endpoints}
\addcontentsline{toc}{section}{Endpoints}

\subsection*{/login}

\subsubsection*{POST}

Loggar in en användare. Förväntar sig en HTTP-body med följande utseende:

\begin{lstlisting}
{
	emailAddress: ANVÄNDARENS EMAILADRESS,
	password: ANVÄNDARENS LÖSENORD
}
\end{lstlisting}

\noindent Returnerar en autentiseringstoken som kan kommas åt antingen via en cookie som sätts
eller själva response-bodyn, som har formatet:

\begin{lstlisting}
{
	authToken: AUTENTISERINGSTOKEN
}
\end{lstlisting}

\noindent Förutsatt att vi lagrar svaret i en variabel kallad \textit{response} kan alltså denna token 
kommas åt som \textit{response.authToken}.

\subsection*{/facebook/login}
\label{sub:/facebook/login}

Loggar in en användare via Facebook.

\paragraph{Headerfält:}
\label{par:headerf_lt_}

\begin{description}
    \item FACEBOOK-AUTH-TOKEN -- en valid autentiseringstoken från Facebooks Graph API
\end{description}

\paragraph{Returvärden:}
\label{par:facebook_login_returv_rden_}

\begin{description}
    \item Vänlista -- en lista innehållandes de av den angivna användarens Facebookvänner som också är registrerade användare av appen.
\end{description}

\subsection*{/swipingsessions}

\subsubsection*{GET}

Hämtar alla swipingsessioner där de angivna användarna deltagit.

\paragraph{Parametrar:}

\begin{itemize}
	
	\item emailAddresses -- en array i json-format innehållandes en serie emailadresser. Matchningen kommer att göras på alla
		emailadresser som skickas, med andra ord kommer bara swipingsessioner där alla användare kopplade till emailaddresserna
		deltagit att hämtas. Matchningen är också exakt i det att den inte heller kommer att göras på de swipingsessioner där
		några andra användare, utöver dessa, deltagit.

\end{itemize}

\paragraph{Exempel:} /swipingsessions?emailAddresses=["user1@example.com","user2@example.com"] 

\subsubsection*{POST}

Skapar en ny swipingsession med alla angivna användare som deltagare. Returnerar den nya swipingsessionens id samt en lista med aktiviteter.

\paragraph{Parametrar:}

\begin{itemize}		
	\item emailAddresses -- en array i json-format innehållandes en serie emailaddresser.			
\end{itemize}		

\paragraph{Returvärden:}
\label{par:swipingsessions_returv_rden_}

\begin{itemize}
    \item swipingSessionId --  Den nya swipingsessionens id, används för att unikt identifiera den vid uppdateringar.
    \item activities -- En lista på aktiviteter som användarna kan swipe:a på.
\end{itemize}

\paragraph{Exempel:} /swipingsessions?emailAddresses=["user1@example.com","user2@example.com"] 

\subsubsection*{PUT}

Uppdaterar en swipingsession med valda aktiviteter för den angivna användaren.

\paragraph{Parametrar:}

\begin{itemize}

	\item swipingSessionId -- id:t för den swipingsession som ska uppdateras. Fås vid skapandet av swipingsessionen eller genom en
		GET-request till denna endpoint.
	\item email -- den emailadress associerad med den användare som just valt aktiviteter.
	\item activities -- en array i json-format innehållandes namnet på de aktiviteter som valts.

\end{itemize}

\paragraph{Exempel:} /swipingsessions?swipingSessionId=1\&email=user1@example.com\&activities=["Moderna Museet","Medelhavsmuseet"]

\section*{Förslag på nya endpoints}
\label{sec:Förslag på nya endpoints}

\subsection*{/accommodation}
\label{sub:/accommodation}

Hämtar hyresobjekt. Främst tänkt för hyresgästen men kan troligtvis användas för hyresvärden för att hämta det objekt som denne hyr ut.

Parametrar:

\begin{itemize}
    \item count
    \item offset
    \item price -- bör stödja olika operatorer som \(=, <, >, <=\) och \(>=\)
    \item size -- bör stödja olika operatorer som \(=, <, >, <=\) och \(>=\)
    \item Olika parametrar baserat på den öppna datan.
\end{itemize}

Returnerar en lista med hyresobjekt.

\subsection*{/tenants}
\label{sub:/tenants}

För hyresvärden. Hämtar de intressenter som swipe:at ja på hyresobjektet.

Parametrar:

\begin{itemize}
    \item renterId
    \item count
    \item offset
\end{itemize}

Returnerar en lista på potentiella hyresgäster.

\subsection*{/matches}
\label{sub:/matches}

För hyresgästen. Hämta de objekt vars hyresvärd swipe:at ja på hyresgästen i fråga.

Parametrar:

\begin{itemize}
    \item tentantId
    \item count
    \item offset
\end{itemize}

Returnerar en lista med hyresobjekt, troligtvis bör också information om hyresvärden följa med. Inte nödvändigtvis för att hyresgästen ska kunna se
den, men för att frontenden ska kunna koppla ihop hyresgäst och hyresvärd för chatt.

\end{document}
