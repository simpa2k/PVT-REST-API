\documentclass{article}

\usepackage[utf8]{inputenc}
\usepackage[swedish]{babel}
\usepackage{listings}
\usepackage[hidelinks]{hyperref} 

\hypersetup{
    linktoc=all
}

\title{API-dokumentation för rentSTHLMs REST-API}
\author{Simon Olofsson}
\date{}

\begin{document}

\lstset{showspaces=false,showstringspaces=false,literate={Ä}{{\"A}}1{Ö}{{\"O}}1}

\maketitle

\newpage

\tableofcontents

\newpage

\section {Endpoints}
\label{sec:Endpoints}

\subsection{/facebook/login}
\label{sub:_facebook_login}

\subsubsection {POST}
\label{ssub:POST}

För inloggning av användare.

Body-parametrar:

\begin{itemize}
    \item token -- en Facebooktoken.
\end{itemize}

Returnerar en applikationsspecifik autentiseringstoken som används vid vidare kommunikation med servern.

\subsection{/users}
\label{sub:_user}

\subsubsection{GET}
\label{ssub:GET}

Hämtar en användare.

Parametrar:

\begin{itemize}
    \item id -- användarens id.
\end{itemize}

Returnerar användarens profil.

\subsubsection {PUT}
\label{ssub:PUT}

Används för att spara profilinformation om en användare. Eftersom användaren tekniskt sett redan är skapad vid inloggning blir all uppdatering av
dennes profil en uppdatering av användaren.

URL-parametrar:

\begin{itemize}
    \item id
    \item description
    \item age
    \item numberoftenants
    \item income
\end{itemize}

Body-parametrar:

\begin{itemize}
    \item images
\end{itemize}

Returnerar den uppdaterade användaren.

\subsection {/accommodation}
\label{sub:/accommodation}

\subsubsection {GET}
\label{ssub:GET}

Hämtar hyresobjekt. Främst tänkt för hyresgästen men kan troligtvis användas för hyresvärden för att hämta det objekt som denne hyr ut.

URL-parametrar:

\begin{itemize}
    \item id
    \item count
    \item offset
    \item price -- bör stödja olika operatorer som \(=, <, >, <=\) och \(>=\)
    \item size -- bör stödja olika operatorer som \(=, <, >, <=\) och \(>=\)
    \item Olika parametrar baserat på den öppna datan.
\end{itemize}

Returnerar en lista med hyresobjekt.

\subsubsection {POST}
\label{ssub:POST}

Skapar ett hyresobjekt.

Body-parametrar:

\begin{itemize}
    \item price
    \item size
    \item location
    \item bilder
\end{itemize}

\subsubsection {PUT}
\label{ssub:PUT}

Uppdaterar ett hyresobjekt.

URL-parametrar:

\begin{itemize}
    \item id
\end{itemize}

Body-parametrar:

\begin{itemize}
    \item price
    \item size
    \item location
    \item images 
\end{itemize}

Returnerar det uppdaterade hyresobjektet.

\subsubsection {DELETE}
\label{ssub:DELETE}

Tar bort ett hyresobjekt.

URL-parametrar:

\begin{itemize}
    \item id
\end{itemize}

\subsection {/tenants}
\label{sub:/tenants}

\subsubsection {GET}
\label{ssub:GET}

För hyresvärden. Hämtar de intressenter som swipe:at ja på hyresobjektet.

URL-parametrar:

\begin{itemize}
    \item renterId
    \item count
    \item offset
\end{itemize}

Returnerar en lista på potentiella hyresgäster.

\subsection {/matches}
\label{sub:/matches}

\subsubsection {GET}
\label{ssub:GET}

Hämta de objekt vars hyresvärd swipe:at ja på hyresgästen i fråga eller de hyresgäster en hyresvärd swipe:at ja på.

Parametrar:

\begin{itemize}
    \item userId 
    \item count
    \item offset
\end{itemize}

Returnerar en lista med hyresobjekt eller en lista med hyresgäster, beroende på vem som gjort requesten. Med listan av hyresobjekt bör troligtvis också
information om hyresvärden följa. Inte nödvändigtvis för att hyresgästen ska kunna se den, men för att frontenden ska kunna koppla ihop hyresgäst och
hyresvärd för chatt.

\subsubsection {POST}
\label{ssub:POST}

Skapar en matchning. Bör troligtvis också kunna hantera att skicka flera matchningar på en gång, så att inte en request behöver göras varje gång en
matchning görs.

Body-parametrar:

\begin{itemize}
    \item tenantId
    \item renterId
    \item accommodationId -- om det ska vara möjligt att lägga upp flera objekt.
\end{itemize}

\subsubsection {DELETE}
\label{ssub:DELETE}

Ångra en matchning.

Url-parametrar:

\begin{itemize}
    \item tenantId
    \item renterId
    \item accommodationId -- om det ska vara möjligt att lägga upp flera objekt.
\end{itemize}

\end{document}
